% 二次形式・ヘッシャン・シルベスター判定法のまとめ
\documentclass\[11pt]{article}
\usepackage{amsmath,amssymb,amsthm,amsfonts}
\usepackage{geometry}
\geometry{margin=25mm}
\title{二次形式と凸性――ヘッシャンからシルベスター判定法へ}
\author{}
\date{}
\begin{document}
\maketitle

\section{二次形式 \$q(x)\$ と凹凸(1 変数)}
1 変数の二次形式は

$$
  q(x)=ax^{2}\qquad(a\in\mathbb R).
$$

\begin{itemize}
\item \$a>0;\Rightarrow;\textbf{下に凸}\$(convex)
\item \$a<0;\Rightarrow;\textbf{上に凸}\$(concave)
\end{itemize}
判断基準は 2 階微分

$$
  q''(x)=2a
$$

の符号のみである。

\section{多変数二次形式と固有値}
\$n\$ 変数ベクトル \$\mathbf x\in\mathbb R^{n}\$ に対し

$$
  q(\mathbf x)=\mathbf x^{\mathsf T}A\mathbf x,\qquad A=A^{\mathsf T}\in\mathbb R^{n\times n}.
$$

\begin{center}
\begin{tabular}{|c|c|c|}
\hline
\$A\$ の固有値 & \$q\$ の形状 & 凹凸 \\
\hline
すべて \$>0\$ & お椀型     & \textbf{下に凸} \\
すべて \$<0\$ & 逆お椀型   & \textbf{上に凸} \\
正負が混在 & 鞍型       & \textbf{不定(非凸・非凹)} \\
\hline
\end{tabular}
\end{center}

\section{ヘッシャン行列と局所形状}
一般の関数 \$f:\mathbb R^{n}\to\mathbb R\$ を点 \$\mathbf x\_{0}\$ 近傍で 2 次までテイラー展開すると
\begin{align\*}
f(\mathbf x) &\approx f(\mathbf x\_{0})
+\nabla f(\mathbf x\_{0})^{\mathsf T}(\mathbf x-\mathbf x\_{0})\\
&\quad+\frac12(\mathbf x-\mathbf x\_{0})^{\mathsf T}H(\mathbf x\_{0})(\mathbf x-\mathbf x\_{0}),
\end{align\*}
ここで \emph{ヘッシャン行列} は

$$
  H(\mathbf x_{0})
  =\bigl[\partial^{2}f/\partial x_{i}\partial x_{j}(\mathbf x_{0})\bigr]_{i,j}
$$

であり,対称行列である.二次項

$$
  q_{H}(\mathbf h)=\mathbf h^{\mathsf T}H(\mathbf x_{0})\,\mathbf h
$$

の符号性によって局所的な凹凸が決まる.

\section{シルベスター判定法(Sylvester's Criterion)}
ヘッシャン \$H\$ の固有値を直接計算せずに,主座小行列式を用いて正定値/負定値を判定する方法である.

\subsection\*{定義}
\$H\_{k}\$ を \$H\$ の左上 \$k\times k\$ ブロック(主座部分行列),\$\Delta\_{k}=\det H\_{k}\$ とする.

\subsection\*{判定条件}
\begin{enumerate}
\item \textbf{正定値}(positive definite)\quad
\$\Delta\_{1}>0,;\Delta\_{2}>0,;\dots,;\Delta\_{n}>0\$.
\item \textbf{負定値}(negative definite)\quad
\$(-1)^{k}\Delta\_{k}>0\quad(k=1,2,\dots,n)\$.
\item 上記いずれにも当てはまらない場合,\textbf{不定値}(indefinite).
\end{enumerate}
半正定値・半負定値の判定では \$\Delta\_{k}\ge0\$(または交互に符号を変えて \$\ge0\$)とするが,\$\Delta\_{k}=0\$ のみでは真の判別ができない場合があるので注意.

\section{例:2 変数関数}

$$
  f(x,y)=-x^{2}-3y^{2}+2xy.
$$

このとき

$$
  H=\begin{pmatrix}-2 & 2\\ 2 & -6\end{pmatrix},\quad
  \Delta_{1}=\det[-2]=-2,\quad
  \Delta_{2}=\det H=(-2)(-6)-2\times2=8.
$$

負定値判定:

$$
  (-1)^{1}\Delta_{1}=2>0,\quad(-1)^{2}\Delta_{2}=8>0.
$$

よって \$H\$ は負定値.従って \$f\$ は局所的に \textbf{上に凸},点 \$(0,0)\$ は局所最大点である.

\section{まとめ}
\begin{itemize}
\item 二次形式の固有値の符号が局所的な凹凸を支配する.
\item ヘッシャン行列は一般関数の多変数 2 階微分として,局所形状を二次形式で表す.
\item ヘッシャンの主座小行列式の符号列により,正定値・負定値・不定値を簡便に判定できる(シルベスター判定法).
\end{itemize}

\end{document}
